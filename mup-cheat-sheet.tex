\documentclass{article}
\usepackage[a4paper, top=0.25em, bottom=3.75em, left=2.5em, right=2.5em]{geometry}
\usepackage{amsmath}  % align environment
\usepackage{amsfonts}  % mathbb
\usepackage{array}  % table formatting
\usepackage{booktabs}  % top/mid/bottomrule
\usepackage{nicefrac}
\usepackage{colortbl} % For table cell background color
\usepackage[table]{xcolor} % Extended color support
\usepackage{multirow}
\usepackage{multicol}
\usepackage{enumitem}  % [leftmargin=*]
\usepackage{pifont} % Provides tick and cross symbols
\usepackage{fancyhdr}  % header
\pagestyle{fancy}  % header
\pagenumbering{gobble}  % No page numbers

\usepackage{etoolbox}

% Patch to fix colortbl and amsmath conflict
% Taken from: https://tex.stackexchange.com/questions/86211/colortbl-and-amsmath-conflict
%%%%%%%%%% HERE COMES THE PATCH %%%%%%%%%%
\makeatletter
\pretocmd\start@align
{%
  \let\everycr\CT@everycr
  \CT@start
}{}{}
\apptocmd{\endalign}{\CT@end}{}{}
\makeatother
%%%%%%%%%% END PATCH %%%%%%%%%%


\renewcommand{\headrulewidth}{0pt}  % no header line
\renewcommand{\footrulewidth}{0.4pt}  % footer line

\fancyfoot[L]{Author: Charlie Blake, 2025}  % Left-aligned header
\fancyfoot[R]{Contact: thecharlieblake @ gmail.com}  % Center-aligned header
% \fancyfoot[R]{GitHub: TODO}  % Right-aligned header

\definecolor{lightblue}{HTML}{a6cee3}
\definecolor{darkblue}{HTML}{1f78b4}
\definecolor{lightgreen}{HTML}{b2df8a}
\definecolor{darkgreen}{HTML}{33a02c}
\definecolor{lightorange}{HTML}{fdbf6f}
\definecolor{darkorange}{HTML}{ff7f00}

\newcommand{\timeseq}{\mathrel{\times}=}
\newcommand{\divideeq}{\mathrel{\div}=}

\newcommand{\tick}{\ding{51}}
\newcommand{\cross}{\ding{55}}

\begin{document}
\begin{center}
    \rule{20em}{0.001em}
    \\
    \vspace{1.5em}
    {\Large The µP Cheat Sheet}
    \vspace{0.5em}
    \\
    \rule{20em}{0.001em}
    \vspace{-0.5em}
\end{center}

\subsection*{What is a parametrization?}

A parametrization is a set of rules that determine the values of three types of hyperparameter (HP) in a model:

\vspace{1em}
\noindent
\begin{tabular}{@{}p{19em}@{}p{3.5em}p{13em}p{13em}}  % Define column widths
    \vspace{-1em}
    \begin{enumerate}
        \item parameter multipliers ($\alpha$)
        \item initialization standard deviations ($\sigma$)
        \item learning rates ($\eta$)
    \end{enumerate}
    &
    \centering \vspace{1.4em} such that
    &
    \textit{for width-parametrizations}
    \centering
    \noindent
    {\begin{align*}
        W_t &= \alpha_W \cdot w_t\\
        w_0 &= \mathcal{N}(0, \sigma_W^2)\\
        w_{t+1} &= w_t - \eta_W \cdot [\textrm{update}]
    \end{align*}}
    &
    \textit{for depth-parametrizations}
    \centering
    \noindent
    {\begin{align*}
        \mathrm{residual\text{-}block}(x) =\\
            \alpha_{\mathrm{res}} \cdot f_{\mathrm{res}}(x) + x
    \end{align*}}
    and $\sigma_{\mathrm{res}}, \eta_{\mathrm{res}}$ apply to all $W$on the residual branch
\end{tabular}

\vspace{-0.5em}

\noindent
These rules are typically functions of the width ($d$) and/or layers ($\ell$) of the model.

\vspace{0.5em}

\noindent
Rather than defining rules for individual $W$s, parameterizations tend to define rules according to three tensor types:

\vspace{-1em}
\noindent
\begin{table}[ht]
    \centering
    \begin{tabular}{@{}p{7em}@{}p{5em}p{5.5em}p{19em}}
        \toprule
        tensor type & fan-in($W$) & fan-out($W$) & example\\
        \midrule
        1. Input & $\Theta(1)$ & $\Theta(d)$ & encoder, embedding, bias, norm params\\
        2. Hidden & $\Theta(d)$ & $\Theta(d)$ & linear layer, convolution\\
        3. Output & $\Theta(d)$ & $\Theta(1)$ & decoder, readout\\
        (Residual) & & & (any parameter tensor on a residual branch)\\
        \bottomrule
    \end{tabular}
\end{table}

\vspace{-1.5em}

\subsection*{Definitions of key parametrizations}

The following table-pair captures important width-parametrizations from the literature. To derive a parametrization, select its column from the left table and take the entries from the right table of the corresponding color:

\def\arraystretch{1.5}%  1 is the default, change whatever you need

\setlength{\aboverulesep}{0pt} % Space above \rules
\setlength{\belowrulesep}{0pt} % Space below \rules

\vspace{1em}
\noindent
% \hspace{-1.25em}
\begin{minipage}[t]{0.45\textwidth}
    \centering
    \begin{tabular}{ccccccc}
        \toprule
        Param. feature & SP & NTP-na & NTP-fa & µP-na & µP & u-µP \\
        \midrule
        down-scale $\eta_{\textrm{in}}$ & \textbf{n/a} & \cellcolor{lightblue}{\cross} & \cellcolor{lightblue}{\cross} & \cellcolor{lightblue}{\cross} & \cellcolor{lightblue}{\cross} & \cellcolor{darkblue}{\textcolor{white}{\tick}} \\
        `full-alignment' & \textbf{n/a} & \cellcolor{lightgreen}{\cross} & \cellcolor{darkgreen}{\textcolor{white}{\tick}} & \cellcolor{lightgreen}{\cross} & \cellcolor{darkgreen}{\textcolor{white}{\tick}} & \cellcolor{darkgreen}{\textcolor{white}{\tick}} \\
        down-scale $\sigma_{\textrm{out}}$ & \cellcolor{lightorange}{\cross} & \cellcolor{lightorange}{\cross} & \cellcolor{lightorange}{\cross} & \cellcolor{darkorange}{\textcolor{white}{\tick}} & \cellcolor{darkorange}{\textcolor{white}{\tick}} & \cellcolor{darkorange}{\textcolor{white}{\tick}} \\
        \bottomrule
    \end{tabular}
\end{minipage}%
\hspace{6em}
\begin{minipage}[t]{0.45\textwidth}
    \centering
    \begin{tabular}{rcccccc}
        \toprule
        \multirow{2}{*}{HP\hspace{1.4em}} & \multicolumn{6}{c}{tensor type}\\ \cline{2-7}
         & \multicolumn{2}{c}{input} & \multicolumn{2}{c}{hidden} & \multicolumn{2}{c}{output}\\
        \midrule
        $\sigma$ & \multicolumn{2}{c}{$1$} & \multicolumn{2}{c}{$\nicefrac{1}{\sqrt{d}}$} & \cellcolor{lightorange}$\nicefrac{1}{\sqrt{d}}$ & \cellcolor{darkorange}{\textcolor{white}{$\;\nicefrac{1}{d}\;$}} \\
        (Adam) $\eta$ & \cellcolor{lightblue}{$\,\,\; 1 \,\,\;$} & \cellcolor{darkblue}{\textcolor{white}{$\nicefrac{1}{\sqrt{d}}$}} & \cellcolor{lightgreen}$\nicefrac{1}{\sqrt{d}}$ & \cellcolor{darkgreen}{\textcolor{white}{$\;\nicefrac{1}{d}\;$}} & \cellcolor{lightgreen}$\nicefrac{1}{\sqrt{d}}$ & \cellcolor{darkgreen}{\textcolor{white}{$\;\nicefrac{1}{d}\;$}} \\
        \bottomrule
    \end{tabular}
\end{minipage}
\vspace{0.75em}

\noindent
No values are given for $\alpha$s due to \textbf{abc-symmetry}, which states that models are invariant to changes of $\alpha, \sigma, \eta$ of the form $\alpha_W \timeseq \theta, \sigma_W \divideeq \theta, \eta_W \divideeq \theta$ for a given $W, \theta$. Different values of $\theta$ define different parametrizations in the same \textit{equivalence class}. $\theta$ is always chosen above such that $\alpha=1$ for the sake of comparison, though other forms are used in the literature (e.g. the Mean Field Parametrization [todo] is in the same equivalence class as µP, but usually presented differently). The only parametrization which specifies a preferred form is u-µP, which uses $\sigma=1$ for numerical stability. Also note that:

\noindent
\begin{multicols}{3} % Specify 3 columns
    \begin{itemize}[leftmargin=*]
        \item $\sigma, \eta$ are only \textit{proportional} to the values in the right table—the constant of proportionality is a tunable HP.
        \item The above $\eta$s apply to all \textbf{optimizers} which guarantee $\Theta(1)$-sized updates (e.g. Adam, Shampoo, Muon). Table 1 of [todo] shows adjustments for other optimizers.
        \item Standard Parametrization (\textbf{SP})'s rules for $\eta$ are presented inconsistently in the literature. Hence they are dropped entirely here.
        \item \textbf{`fa/na'} denotes full/no-alignment assumptions from [todo]
        \item The typical presentation of Neural Tangent Parametrization (\textbf{NTP}) is different to the one shown here (see Table 1 [tp4]) and has been shown to scale poorly. The `fa/na' variants use more appropriate $\eta$s.
        \item \textbf{u-µP}'s down-scaled $\eta_{\textrm{in}}$ has only been validated on embedding-style input layers.
    \end{itemize}
\end{multicols}

TODO: something on depth
% depth

\subsection*{F.A.Q.}
\vspace{-1em}

\noindent
\begin{multicols}{3} % Start a three-column environment

    \noindent
    \textit{What are parametrizations trying to do?}\vspace{0.4em}\\
    This is some text that will appear in the first column.\\
    \columnbreak

    \noindent
    \textit{What parametrization should I use?}\vspace{0.4em}\\
    This is text for the second column.\\
    \columnbreak

    \noindent
    \textit{Any practical tips for applying this?}\vspace{0.4em}\\
    This is text for the third column.
    
    \end{multicols}

\end{document}
